\documentclass[]{article}
\usepackage{lmodern}
\usepackage{amssymb,amsmath}
\usepackage{ifxetex,ifluatex}
\usepackage{fixltx2e} % provides \textsubscript
\ifnum 0\ifxetex 1\fi\ifluatex 1\fi=0 % if pdftex
  \usepackage[T1]{fontenc}
  \usepackage[utf8]{inputenc}
\else % if luatex or xelatex
  \ifxetex
    \usepackage{mathspec}
  \else
    \usepackage{fontspec}
  \fi
  \defaultfontfeatures{Ligatures=TeX,Scale=MatchLowercase}
\fi
% use upquote if available, for straight quotes in verbatim environments
\IfFileExists{upquote.sty}{\usepackage{upquote}}{}
% use microtype if available
\IfFileExists{microtype.sty}{%
\usepackage{microtype}
\UseMicrotypeSet[protrusion]{basicmath} % disable protrusion for tt fonts
}{}
\usepackage[margin=1in]{geometry}
\usepackage{hyperref}
\hypersetup{unicode=true,
            pdftitle={1.1 Preliminaries and Software installation},
            pdfauthor={Adapted from Software Carpentry; modified by Maina,Stephanie},
            pdfborder={0 0 0},
            breaklinks=true}
\urlstyle{same}  % don't use monospace font for urls
\usepackage{color}
\usepackage{fancyvrb}
\newcommand{\VerbBar}{|}
\newcommand{\VERB}{\Verb[commandchars=\\\{\}]}
\DefineVerbatimEnvironment{Highlighting}{Verbatim}{commandchars=\\\{\}}
% Add ',fontsize=\small' for more characters per line
\usepackage{framed}
\definecolor{shadecolor}{RGB}{248,248,248}
\newenvironment{Shaded}{\begin{snugshade}}{\end{snugshade}}
\newcommand{\KeywordTok}[1]{\textcolor[rgb]{0.13,0.29,0.53}{\textbf{#1}}}
\newcommand{\DataTypeTok}[1]{\textcolor[rgb]{0.13,0.29,0.53}{#1}}
\newcommand{\DecValTok}[1]{\textcolor[rgb]{0.00,0.00,0.81}{#1}}
\newcommand{\BaseNTok}[1]{\textcolor[rgb]{0.00,0.00,0.81}{#1}}
\newcommand{\FloatTok}[1]{\textcolor[rgb]{0.00,0.00,0.81}{#1}}
\newcommand{\ConstantTok}[1]{\textcolor[rgb]{0.00,0.00,0.00}{#1}}
\newcommand{\CharTok}[1]{\textcolor[rgb]{0.31,0.60,0.02}{#1}}
\newcommand{\SpecialCharTok}[1]{\textcolor[rgb]{0.00,0.00,0.00}{#1}}
\newcommand{\StringTok}[1]{\textcolor[rgb]{0.31,0.60,0.02}{#1}}
\newcommand{\VerbatimStringTok}[1]{\textcolor[rgb]{0.31,0.60,0.02}{#1}}
\newcommand{\SpecialStringTok}[1]{\textcolor[rgb]{0.31,0.60,0.02}{#1}}
\newcommand{\ImportTok}[1]{#1}
\newcommand{\CommentTok}[1]{\textcolor[rgb]{0.56,0.35,0.01}{\textit{#1}}}
\newcommand{\DocumentationTok}[1]{\textcolor[rgb]{0.56,0.35,0.01}{\textbf{\textit{#1}}}}
\newcommand{\AnnotationTok}[1]{\textcolor[rgb]{0.56,0.35,0.01}{\textbf{\textit{#1}}}}
\newcommand{\CommentVarTok}[1]{\textcolor[rgb]{0.56,0.35,0.01}{\textbf{\textit{#1}}}}
\newcommand{\OtherTok}[1]{\textcolor[rgb]{0.56,0.35,0.01}{#1}}
\newcommand{\FunctionTok}[1]{\textcolor[rgb]{0.00,0.00,0.00}{#1}}
\newcommand{\VariableTok}[1]{\textcolor[rgb]{0.00,0.00,0.00}{#1}}
\newcommand{\ControlFlowTok}[1]{\textcolor[rgb]{0.13,0.29,0.53}{\textbf{#1}}}
\newcommand{\OperatorTok}[1]{\textcolor[rgb]{0.81,0.36,0.00}{\textbf{#1}}}
\newcommand{\BuiltInTok}[1]{#1}
\newcommand{\ExtensionTok}[1]{#1}
\newcommand{\PreprocessorTok}[1]{\textcolor[rgb]{0.56,0.35,0.01}{\textit{#1}}}
\newcommand{\AttributeTok}[1]{\textcolor[rgb]{0.77,0.63,0.00}{#1}}
\newcommand{\RegionMarkerTok}[1]{#1}
\newcommand{\InformationTok}[1]{\textcolor[rgb]{0.56,0.35,0.01}{\textbf{\textit{#1}}}}
\newcommand{\WarningTok}[1]{\textcolor[rgb]{0.56,0.35,0.01}{\textbf{\textit{#1}}}}
\newcommand{\AlertTok}[1]{\textcolor[rgb]{0.94,0.16,0.16}{#1}}
\newcommand{\ErrorTok}[1]{\textcolor[rgb]{0.64,0.00,0.00}{\textbf{#1}}}
\newcommand{\NormalTok}[1]{#1}
\usepackage{graphicx,grffile}
\makeatletter
\def\maxwidth{\ifdim\Gin@nat@width>\linewidth\linewidth\else\Gin@nat@width\fi}
\def\maxheight{\ifdim\Gin@nat@height>\textheight\textheight\else\Gin@nat@height\fi}
\makeatother
% Scale images if necessary, so that they will not overflow the page
% margins by default, and it is still possible to overwrite the defaults
% using explicit options in \includegraphics[width, height, ...]{}
\setkeys{Gin}{width=\maxwidth,height=\maxheight,keepaspectratio}
\IfFileExists{parskip.sty}{%
\usepackage{parskip}
}{% else
\setlength{\parindent}{0pt}
\setlength{\parskip}{6pt plus 2pt minus 1pt}
}
\setlength{\emergencystretch}{3em}  % prevent overfull lines
\providecommand{\tightlist}{%
  \setlength{\itemsep}{0pt}\setlength{\parskip}{0pt}}
\setcounter{secnumdepth}{0}
% Redefines (sub)paragraphs to behave more like sections
\ifx\paragraph\undefined\else
\let\oldparagraph\paragraph
\renewcommand{\paragraph}[1]{\oldparagraph{#1}\mbox{}}
\fi
\ifx\subparagraph\undefined\else
\let\oldsubparagraph\subparagraph
\renewcommand{\subparagraph}[1]{\oldsubparagraph{#1}\mbox{}}
\fi

%%% Use protect on footnotes to avoid problems with footnotes in titles
\let\rmarkdownfootnote\footnote%
\def\footnote{\protect\rmarkdownfootnote}

%%% Change title format to be more compact
\usepackage{titling}

% Create subtitle command for use in maketitle
\newcommand{\subtitle}[1]{
  \posttitle{
    \begin{center}\large#1\end{center}
    }
}

\setlength{\droptitle}{-2em}
  \title{1.1 Preliminaries and Software installation}
  \pretitle{\vspace{\droptitle}\centering\huge}
  \posttitle{\par}
  \author{Adapted from Software Carpentry; modified by Maina,Stephanie}
  \preauthor{\centering\large\emph}
  \postauthor{\par}
  \predate{\centering\large\emph}
  \postdate{\par}
  \date{13th of March 2018}


\begin{document}
\maketitle

\textbf{How this is going to go down}

We are going to cover the material in a different way than you would get
taught a statistics course, and a different way than you would get
taught a programming course.

We're going to start pretty slow, and slowly ramp the pace up. The idea
is to get you to a point where you can continue learning yourself. It is
not possible to learn to program in three days. It will take time and
practice.

If you get lost in the material at the second half of the day, don't
despair, make sure you speak up.

We're going to focus on concepts, best practices and workflow almost as
much on getting you going with the syntax and commands. R is simply too
large to teach in three days that the full course runs for. Our main
hope is that you get enough of a flavour of it to continue the learning
process yourself. We probably have more material here than we'll get
through today, and we feel that this really just scratches the surface.

\textbf{Learning R can be frustrating}

Learning R is not necessarily hard, but not necessarily easy either.
Different people will make the logical connections faster than others,
and until it ``clicks'' it may seem like a battle. Ask questions as you
are unclear and we'll help make those connections faster. It is a
programming language don't think of it a statistical program that you
use from a command line think of it as a programming language that
happens to have a lot of statistical functions. The big challenge is
going to be bridging between the nebuluous ``what I think I want to do''
and the precice ``computer - this is what I need you to do''. It is easy
to underestimate how precice instructions need to be. We'll see
examples, but things like (``take the mean of the leg lengths by
species'' are intuitively obvious to us, but can be hard precicely
convey to a computer. what if there are missing data? does it matter if
there are different numbers individual per species? does it matter if
there are different sized legs on each species (which leg?) what happens
if someone gave ranges for some of the leg measurements, rather than one
number? what about subspecies? There is a good Douglas Adams quote about
this: ``If you really want to understand something, the best way is to
try and explain it to someone else. That forces you to sort it out in
your mind. And the more slow and dim-witted your pupil, the more you
have to break things down into more and more simple ideas. And that's
really the essence of programming. By the time you've sorted out a
complicated idea into little steps that even a stupid machine can deal
with, you've learned something about it yourself.'' (from Dirk Gentley's
Holistic Detective Agency).

\textbf{INSTALLING R and R Studio}

R and R Studio are separate packages. You will need to install R first.
R is the basic package we are using. R Studio is an add-on that make R
easier to use for beginners. These instructions should work for Windows
and MAC users for installing R and R Studio.

\paragraph{\texorpdfstring{\textbf{1.INSTALLING
R:}}{1.INSTALLING R:}}\label{installing-r}

\begin{itemize}
\item
  Go to \url{http://www.r-project.org/}

  \begin{itemize}
  \tightlist
  \item
    on the ``Getting Started'' box, click on ``download R.''
  \item
    on \url{https://cran.r-project.org/mirrors.html}, find a site of
    your choice, ideally the closest to your actual location.
  \item
    Click to go to that site
  \end{itemize}
\item
  On the mirror site of your choice, click on your operating system:

  \begin{itemize}
  \tightlist
  \item
    Download R for Linux
  \item
    Download R for (Mac) OS X
  \item
    Download R for Windows
  \end{itemize}
\item
  If \textbf{Windows}:

  \begin{itemize}
  \item
    click on ``\textbf{base}'' and then on \textbf{Download R 3.4.1 for
    Windows}.
  \item
    save the file ``\textbf{R-3.4.1-win.exe}'' and then click on it to
    execute it.
  \item
    Once the dialog box opens, click ``\textbf{RUN}.'' A Setup Wizard
    should appear.
  \item
    Keep clicking ``\textbf{Next}'' (or change features if you
    understand them), until it is finished.

    \textbf{Note} that 3.4.1 is the current version as this is being
    written, but use whatever shows up as current.
  \end{itemize}
\item
  If \textbf{Mac}:

  \begin{itemize}
  \tightlist
  \item
    click on the last version of the R package \textbf{R-3.4.1.pkg} on
    the page \textbf{R for Mac OS X} under the \textbf{Files} section
  \item
    Save the \textbf{.pkg file}
  \item
    Double-click it to open, and follow the installation instructions.
  \end{itemize}
\end{itemize}

Now that \textbf{R} is installed, you should now see an icon on your
desktop, with a large capital ``\textbf{R}''.

You then need to download and install RStudio.

\paragraph{\texorpdfstring{\textbf{2. INSTALLING R
STUDIO:}}{2. INSTALLING R STUDIO:}}\label{installing-r-studio}

We're going to focus on using RStudio for two reasons:

It makes my life easier: you are all using the same tools and we'll be
able to help you with your problems faster.

It will make your life easier: it's got lots of features that help
people, especially beginners. It will help you organise your work,
develop good workflows. On the other hand, it's not very intrusive and
if you use a different interface (such as the plain R interface that you
installed) it will feel very similar.

\begin{itemize}
\tightlist
\item
  Go to \url{http://www.rstudio.com}
\item
  click on ``Download RStudio'' and follow the directions for your
  operating system.
\end{itemize}

\textbf{USING R AND R STUDIO} Open R Studio by clicking on the icon.
You're ready to go!

\textbf{Getting started with RStudio}

Load R studio however you do that on your platform.

RStudio has four panes:

\begin{itemize}
\item
  \begin{enumerate}
  \def\labelenumi{\arabic{enumi}.}
  \tightlist
  \item
    Bottom left - this is the R interpreter. If you type code here, it
    is ``evaulated'' so that you get an answer.
  \end{enumerate}
\item
  \begin{enumerate}
  \def\labelenumi{\arabic{enumi}.}
  \setcounter{enumi}{1}
  \tightlist
  \item
    Top left - editor for writing longer pieces of code.
  \end{enumerate}
\item
  \begin{enumerate}
  \def\labelenumi{\arabic{enumi}.}
  \setcounter{enumi}{2}
  \tightlist
  \item
    Top right will tell you things about objects in the workspace. We'll
    get to this soon, but this will be things like data objects, or
    functions that will process them. It is completely unrelated to the
    file system. The ``History'' tab will keep an eye on what you've
    done.
  \end{enumerate}
\item
  \begin{enumerate}
  \def\labelenumi{\arabic{enumi}.}
  \setcounter{enumi}{3}
  \tightlist
  \item
    Will display files, plots, packages, and help information, usually
    as needed. We'll do everything in a project, as that will help when
    we get some data.
  \end{enumerate}
\item
  ``Project'': ``Create Project.''
\item
  choose ``New Project, (start a project in a new directory)''.
\item
  Leave the ``Type'' as the default.
\item
  In the ``Directory name'' type the name for the project (in our case
  \texttt{g2g} might be a good name).
\item
  In the ``Create project as a subdirectory of'' field select (type or
  browse) for the parent directory of the project. By default this is
  probably your home directory, but you might prefer your Documents
  folder. The RStudio window morphs around a bit, and the top left pane
  will disappear.
\end{itemize}

In the bottom right panel, make sure that the ``Files'' tab is selected
and make a new folder called \texttt{data}. I strongly recommend keeping
a directory like this in every project that contains your data. Treat it
read only (that is, write once, then basically don't change). This may
be a large shift in thinking if you've come from doing data analysis and
management in Excel.

In more complicated projects, I would generally have a folder called
\texttt{R} that contains scripts and function definitions, another
called \texttt{figs} that contained figures that I've generated, and one
called \texttt{doc} that contains the manuscript, talks, etc.

We're going to spend a bit of time using a data set. You can download
this from here and put it into that directory. Download this file, open
the \texttt{g2g/data} folder and move it there (if you click More: show
folder in new window, you'll get a file browser window opening in about
the right place). Similarly, also grab
\href{https://nicercode.github.io/intro/data/seed_root_herbivores.txt}{this
file} and put it in the \texttt{data} directory too.

To make sure everything is working properly, in the console window type.

\begin{Shaded}
\begin{Highlighting}[]
\KeywordTok{file.exists}\NormalTok{(}\StringTok{"data/seed_root_herbivores.csv"}\NormalTok{)}
\end{Highlighting}
\end{Shaded}

which should print

\begin{verbatim}
## [1] FALSE
\end{verbatim}

\section{3. EXITING R AND R STUDIO}\label{exiting-r-and-r-studio}

You can exit R and R Studio by typping q() at the command prompt. Note
that this is the letter q followed by open and closed parentheses.

\begin{Shaded}
\begin{Highlighting}[]
\KeywordTok{q}\NormalTok{()}
\end{Highlighting}
\end{Shaded}


\end{document}
